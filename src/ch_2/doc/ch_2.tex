\documentclass[10pt]{article}
\usepackage{graphicx}
\usepackage{float}
\usepackage{amsmath}
\usepackage{amsfonts}
\usepackage[brazilian]{babel}
\usepackage[utf8]{inputenc}
\usepackage[backend=biber]{biblatex}
\usepackage{csquotes}
\usepackage[usenames,dvipsnames,svgnames,table]{xcolor}
%\usepackage{docmute}
\usepackage{array}
\usepackage{multicol}
\usepackage{listings}
\usepackage{geometry}
\usepackage[T1]{fontenc}

\addbibresource{ch_2.bib}

\newcommand{\fromeng}[1]{\footnote{do inglês: \textit{#1}}}
\newcommand{\tit}[1]{\textit{#1}}
\newcommand{\tbf}[1]{\textbf{#1}}
\newcommand{\ttt}[1]{\texttt{#1}}

\newcolumntype{C}[1]{>{\centering\let\newline\\\arraybackslash\hspace{0pt}}m{#1}}

\lstset{%
	language=R,
	basicstyle=\scriptsize\ttfamily,
	commentstyle=\ttfamily\color{gray},
	numbers=left,
	numberstyle=\ttfamily\color{blue}\footnotesize,
	stringstyle=\ttfamily\color{olive}\footnotesize,
	stepnumber=1,
	numbersep=5pt,
	backgroundcolor=\color{white},
	showspaces=false,
	showstringspaces=false,
	showtabs=false,
	frame=single,
	tabsize=2,
	captionpos=b,
	breaklines=true,
	breakatwhitespace=false,
	%title=\lstname,
	escapeinside={},
	keywordstyle={},
	morekeywords={}
}

\begin{document}

\begin{titlepage}
	\centering
	{\scshape\Large MC884/MO444 - Aprendizado de Máquina\par}
	\vspace{1.5cm}
	{\huge\bfseries \tit{Support Vector Machines} e validação cruzada\par}
	\vspace{1cm}
	{\itshape Erik de Godoy Perillo - RA135582\par}
	\vfill
	Universidade Estadual de Campinas 
	\vfill
	{\large \today\par}
\end{titlepage}

\newpage

\section{Introdução}
O objetivo do trabalho era experimentar com a técnica de 
\tit{SVM}\fromeng{Support Vector Machines} e sua validação por meio de
\tit{k-folds} com busta de hiperparâmetros por \tit{grid search}.

\subsection{Implementação}
A linguagem de implementação escolhida foi o R.
Todo o código utilizado no relatório encontra-se na seção~\ref{code}.
Ao longo do documento, linhas do código serão citadas para referência no mesmo.
A função \ttt{main} (linha 130 da seção~\ref{code}) executa todos os itens em
ordem, mostrando os resultados.

\section{Enunciado}
\tit{Treine um SVM com kernel RBF nos dados do arquivos.
A validação externa deve ser 5-fold estratificado.
Para cada conjunto de treino da validação externa faça um 3-fold para 
escolher os melhores hiperparametros para C (cost) e $\gamma$ (gamma).
Faça um grid search de para o C nos valores $2^{-5}, 2^{-2}, 2^0, 2^2, 2^5$ 
e gamma nos valores $2^{-15}, 2^{-10}, 2^{-5}, 2^0, 2^5$.}\newline

Os valores especificados para os \tit{folds}, $C$ e $\gamma$
são declarados das linhas 9 a 15 do código na seção~\ref{code}. 
A função que faz o \tit{grid search} está declarada na linha
28 do código. 
Ela faz a procura da melhor combinação de parâmetros $C$ e $\gamma$.

A função que, dados os dados e o número de folds, procura os melhores parâmetros
$C$ e $\gamma$ entre os $k$ possíveis \tit{folds} (fazendo \tit{grid search} 
em cada um deles) é a \ttt{get\_best\_svm\_params}, declarada na linha 57.

Além dessa procura de melhor combinação entre $3$ folds e os hiperparâmetros,
pede-se que esses $3$ folds venham de $5$ folds externos. 
A função que finalmente faz todas essas partes, além de estimar a 
acurácia média do sistema, é declarada como \ttt{mean\_accuracy\_estimate} na
linha 101 do código.

Na função \ttt{main} da linha 130 são feitos os dois passos principais pedidos:
primeiro, a acurácia é estimada na linha 147. Depois, os parâmetros finais
para o sistema são escolhidos na linha 159.

A \tbf{saída} do código todo sendo executado pela \ttt{main} encontra-se
na seção~\ref{output}.

\subsection{Perguntas}
\begin{enumerate}
	\item \tit{Qual a acurácia média na validação de fora?}

		A acurácia média, como indica a linha 43 da saída na seção~\ref{output},
		é de $92.44\%$.	

	\item \tit{Quais os valores de $C$ e $\gamma$ a serem usados no
		classificador final? (fazer 3-fold no conjunto todo)}

		Usando-se o 3-fold, obtivemos, como a linha 56 da saída indica,
		os parâmetros finais:
		$$C = 4,~\gamma = 0.031250$$
\end{enumerate}

\newpage

\section{Código-fonte}
\label{code}
\lstinputlisting{../ch_2.r}

\newpage

\section{Saída do código}
\label{output}
\lstinputlisting{../output.txt}

\printbibliography

\end{document}
